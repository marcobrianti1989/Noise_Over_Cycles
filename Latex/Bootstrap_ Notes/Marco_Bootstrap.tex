\documentclass{article}
\usepackage[utf8]{inputenc}
\usepackage[english]{babel}
\usepackage[a4paper,top=3cm,bottom=3cm,left=3cm,right=3cm,%
bindingoffset=0mm]{geometry}
\usepackage{amssymb}
\usepackage{amsmath}
\newtheorem{prop}{Proposition}
\newtheorem{lemma}{Lemma}
\newenvironment{proof}[1][Proof]{\begin{trivlist}
\item[\hskip \labelsep {\bfseries #1}]}{\end{trivlist}}
\newcommand{\qed}{\nobreak \ifvmode \relax \else
      \ifdim\lastskip<1.5em \hskip-\lastskip
      \hskip1.5em plus0em minus0.5em \fi \nobreak
      \vrule height0.75em width0.75em depth0em\fi}
\usepackage{tikz}
\usepackage{graphicx}
\usepackage{rotating}
\usepackage{float}
\linespread{1.3}
\raggedbottom




%
\font\reali=msbm10 at 12pt
% subsets of real numbers
\newcommand{\real}{\hbox{\reali R}}
\newcommand{\realp}{\hbox{\reali R}_{\scriptscriptstyle +}}
\newcommand{\realpp}{\hbox{\reali R}_{\scriptscriptstyle ++}}
\newcommand{\R}{\mathbb{R}}
\DeclareMathOperator{\E}{\mathbb{E}}
%

\title{Bootstrap}
\author{Marco Brianti}
\date{A.Y. 2018/2019}

\begin{document}
	\large{

\maketitle
As long as $t \geq H$, then the dependent variable of interest $Y_t$ can be represented by $H$ equations where $H$ is the forecast horizon.
\begin{eqnarray}\label{equation:Yt_h0}
Y_t = B_0 U_t + \varepsilon_t^0
\end{eqnarray}
\begin{eqnarray}\label{equation:Yt_h1}
Y_t = B_1 U_{t-1} + \varepsilon_{t-1}^1
\end{eqnarray}
\begin{eqnarray}\label{equation:Yt_h2}
Y_t = B_2 U_{t-2} + \varepsilon_{t-2}^2
\end{eqnarray}
$$
\vdots
$$
\begin{eqnarray}\label{equation:Yt_Hminus1}
Y_t = B_{H-1} U_{t-H+1} + \varepsilon_{t-H+1}^{H-1}
\end{eqnarray}
\begin{eqnarray}\label{equation:Yt_H}
Y_t = B_H U_{t-H} + \varepsilon_{t-H}^H
\end{eqnarray}
Now, lag Equation \ref{equation:Yt_h0} and isolate over $U_{t-1}$ as follows,
\begin{eqnarray}\label{equation:Utminus1}
U_{t-1} = \frac{Y_{t-1} - \varepsilon_{t-1}^0}{B_0}
\end{eqnarray}
Substitute Equation \ref{equation:Utminus1} into Equation \ref{equation:Yt_h1} as follows,
\begin{eqnarray}\label{equation:Yt_Ytminus1}
Y_t = \frac{B_1}{B_0} Y_{t-1} - \frac{B_1}{B_0} \varepsilon_{t-1}^0 + \varepsilon_{t-1}^1
\end{eqnarray}
Now there are two possible avenues:
\begin{enumerate}
	\item Lag twice Equation \ref{equation:Yt_h0}, isolate over $U_{t-2}$, substitute into Equation \ref{equation:Yt_h2} and obtain
	\begin{eqnarray}
	Y_t = \frac{B_2}{B_0} Y_{t-2} - \frac{B_2}{B_0} \varepsilon_{t-2}^0 + \varepsilon_{t-2}^2
	\end{eqnarray}
	As a general result we obtain,
		\begin{eqnarray}
	Y_t = \frac{B_h}{B_0} Y_{t-h} - \frac{B_h}{B_0} \varepsilon_{t-h}^0 + \varepsilon_{t-h}^h
	\end{eqnarray}
	\item Lat Equation \ref{equation:Yt_h1}, isolate over $U_{t-2}$, substitute into Equation \ref{equation:Yt_h2} and obtain 
		\begin{eqnarray}
	Y_t = \frac{B_2}{B_1} Y_{t-1} - \frac{B_2}{B_1} \varepsilon_{t-2}^1 + \varepsilon_{t-2}^2
	\end{eqnarray}
	As a general result we obtain,
	\begin{eqnarray}
	Y_t = \frac{B_h}{B_{h-1}} Y_{t-1} - \frac{B_h}{B_{h-1}} \varepsilon_{t-h}^{h-1} + \varepsilon_{t-h}^h
	\end{eqnarray}
\end{enumerate}




}
\end{document}

