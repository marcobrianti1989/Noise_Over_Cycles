\documentclass[hyperref={pdfpagelabels=false}]{beamer}
\setbeamercolor{background canvas}{bg=white}
\usepackage{graphicx,lmodern,subfigure,ulem,color,graphicx,tikz,booktabs,natbib}
\usepackage{mathrsfs}
\usetheme{Warsaw}
%\definecolor{beamer@blendedblue}{rgb}{0.1,0.5,0.1}
%\definecolor{ForestGreen}{RGB}{60, 140, 60}
%\setbeamercolor{structure}{fg=beamer@blendedblue}
\setbeamertemplate{navigation symbols}{}
\setbeamertemplate{footline}[frame number]
\bibliographystyle{chicago}
\newcommand{\spitem}{\vspace{.3cm}\item}
\newcommand{\elas}{$E_{labor}$}
\def \ourFigPath {../../} 

\usetheme[
outer/progressbar=foot,
outer/numbering=none
]{metropolis}



\title{Putting the Cycle Back into Business Cycle Analysis}
\author{Beaudry, Galizia, and Portier (forthcoming AER)}
\institute{Macro Reading Group, Boston College}
\date{February 2019}


\begin{document}
	
	\frame{
		
		
		\maketitle
}



\frame{\frametitle{Business Cycle Analysis and Modern Macroeconomics}
	
Forces and mechanisms that drive economic fluctuations remain a debated subject. 


\

Two theoretical approaches:
	
	
	\begin{enumerate}
		\item BCs are primarily driven by persistent exogenous shocks
		
		
		\item BCs are mostly driven by forces internal to the economy which endogenously favor recurrent periods of boom and bust.
	\end{enumerate}   

\

This paper argues that data favors the second theoretical approach. 
		
}

\frame{\frametitle{Appeals of the First Approach}
	
	\begin{enumerate}
		\item Empirical estimation of DSGE models support the existence and the persistence of exogenous driving forces to explain the data. 
		
		\
		
		\
		
		\item Since Granger (1966) and Sargent (1987), it has been argued that data are generally not supportive of strong internal boom-bust mechanisms.
	\end{enumerate}
	
		
}

\frame{\frametitle{This Paper}
	
\begin{enumerate}
	\item They examine spectral density properties of macro aggregates
	\begin{itemize}
		\item a recurrent peak in several spectral densities at periodicities around 9 to 10 years
	\end{itemize}

\


	\item They analyze necessary features to theoretically reproduce peaks in the spectral densities
	\begin{itemize}
		\item strategic complementaries across agents
		\item accumulation of a stock with decreasing returns 
	\end{itemize}

\


	\item They estimate a NK model where persistent shocks and endogenous cyclical mechanisms compete to explain data
	\begin{itemize}
	\item estimation favors endogenous mechanisms to match empirical spectral density 
	\item estimation suggests the existence of stochastic limit cycles 
\end{itemize}	
\end{enumerate}
	
	
}
	

\frame{\frametitle{Empirical Evidence - NBER Recessions}
	
\begin{center}
	\includegraphics[scale=0.3]{NBER}
\end{center}
	
}


\frame{\frametitle{Empirical Evidence (II) - Hours Worked per Capita}
	

\begin{center}
	\includegraphics[scale=0.4]{spectrum_four}
\end{center}

}

\frame{\frametitle{Empirical Evidence (II) - Financial Variables}
	
	\begin{center}
		\includegraphics[scale=0.38]{financial}
	\end{center}
	
	
}


\frame{\frametitle{Spectra Implied by Current DSGE Models}
	
	\begin{center}
		\includegraphics[scale=0.34]{DSGE_fail}
	\end{center}
	

}






\frame{\frametitle{Takeaway}

\begin{itemize}
	\item Probability to fall in a recession rises every 8-10 years
	\item Measures of labor activity exhibit significant spectral peaks at a periodicity of around 36-40 quarters
	\item Financial variables display analogous patters
	\item Current theoretical models are unable to match previous facts
	\end{itemize}

\

They provide a class of models able to reproduce cyclical outcomes based on equilibrium interactions internal to the model.
	

	
}

\frame{\frametitle{A Class of Models}
	
Consider an environment with $N$ agents indexed by $j$.

\

Agent $j$ makes decision $e_{j,t}$ according to,
\begin{eqnarray}
e_{j,t} = \alpha_1 X_{j,t} + \alpha_2 e_{j,t-1} + \alpha_3 q_t + \mu_t, \ \ \ 0 < \alpha_2 < 1
\end{eqnarray}
where is $X_t$ is a stock variable with the following law of motion,
\begin{eqnarray}
X_{j,t+1} = (1 - \delta) X_{j,t} + e_{j,t}, \ \ \ 0 < \delta < 1
\end{eqnarray}
and $\mu_t$ is an exogenous driving force. Finally, $q_t$ is a aggregate market determined variable,
\begin{eqnarray}
q_t = \alpha_4 \frac{1}{N} \sum_j e_{j,t} = \alpha_4 e_{t}
\end{eqnarray}
where $\alpha_3 \alpha_4$ governs the degree of strategic complementarity (substitutability) in the economy.


}


\frame{\frametitle{Spectrum of $e_t$}
	
	Invoke symmetry and solve for $e_t$,
	\begin{eqnarray*}
	e_t = \bigg( \frac{\alpha_1 + \alpha_2}{1 - \alpha_3 \alpha_4} + 1 - \delta \bigg)e_{t-1} - \frac{\alpha_2(1-\delta)}{1 - \alpha_3 \alpha_4} e_{t-2} + \frac{1-(1-\delta)L}{1 - \alpha_3\alpha_4}\mu_t
	\end{eqnarray*}
	which implies the following spectral density
	\begin{eqnarray*}
    s_e(\omega) = s_{\mu}(\omega) \frac{[1 - (1 -\delta)\exp(i\omega)][1 - (1 -\delta)\exp(i\omega)]}{(1 - \alpha_3 \alpha_4)^2}g(\omega)
    \end{eqnarray*}
	where 
	\begin{itemize}
		\item $g(\omega) \equiv [B(\exp(i\omega)) B(\exp(i\omega))]^{-1}$
	    \item $B(L) \equiv 1 - \Big( \frac{\alpha_1 + \alpha_2}{1 - \alpha_3 \alpha_4} + 1 - \delta \Big)L + \frac{\alpha_2(1-\delta)}{1 - \alpha_3 \alpha_4}L^2$
	\end{itemize}
}



\frame{\frametitle{Necessary Conditions for Peak in the Spectral Density}


Sargent (1979): necessary conditions is $B(L)$ to have complex roots in $L$.

\

Assumption: parameters are such that if $\alpha_3 \alpha_4 = 0$, then the eigenvalues are real, positive and smaller than 1.


Then, necessary conditions are
\begin{itemize}
	\item $\alpha_3 \alpha_4 > 0$: strategic complementarity
	\item $\alpha_1 < 0$: decreasing returns in $X_{j,t}$
\end{itemize}

\begin{eqnarray*}
\begin{cases}
e_{j,t} = \alpha_1 X_{j,t} + \alpha_2 e_{j,t-1} + \alpha_3 \alpha_4 e_t + \mu_t \\
X_{j,t+1} = (1 - \delta) X_{j,t} + e_{j,t}
\end{cases}
\end{eqnarray*}

}



\frame{\frametitle{Technical Ingredients}
	
	In order to have a peak in the spectral density which is not driven by exogenous forces, they need 
	
	
	\begin{itemize}
		\item Complex eigenvalues 
		\begin{itemize}
			\item Dynamics are represented by trigonometric functions
			\item For an AR(2) process complex eigenvalues are a necessary condition for a peak in the spectral density (Sargent, 1979)
		\end{itemize}
		
		\
		
		
		\item Local instability surrounded by a stochastic limit cycle 
		\begin{itemize}
			\item Cyclical dynamics are perpetual
			\item Main critique of limit cycle dynamics: predictability and regularity of the cycle
			\item However, when a limit cycle is perturbed by unpredictable disturbances, size and period of the cycle changes permanently
		\end{itemize}
	\end{itemize} 
	

}

\frame{
	\frametitle{Economic Ingredients}
	
		In order to have a peak in the spectral density which is not driven by exogenous forces, they need 
	

	
	\begin{itemize}
		\item Strategic complementarity 
		\begin{itemize}
			\item It is a well-known source of instability
		\end{itemize}
		
		\
		
		
		\item Decreasing return to scale in a stock variable 
				\begin{itemize}
			\item When combined with strategic complementarity, the economy exhibits periods of accumulation and dissipation.
		\end{itemize}
	\end{itemize} 


}

\frame{\frametitle{A New Keynesian Model}









}



\frame{\frametitle{Household}
	
	Household $h$'s preferences are given by
	\begin{eqnarray}
	E_0 \sum_t \beta^t \zeta_{t-1} [U(C_{h,t} - \gamma C_{t-1}) + \nu (1 - e_{h,t})]
	\end{eqnarray}
    In addition to purchase consumption services $C_{h,t}$ at price $P_t$ and labor $e_{h,t}$ at wage $W_t$, household $h$ decides to purchase an amount $I_t$ of durable consumption $X_{h,t}$ at price $P_t^X$.
    
    \
    
    Law of motion of $X_{h,t}$ is: 
    $$
    X_{h,t+1} = (1 - \delta)X_{h,t} + I_t
    $$
    and household budget constraint is: 
    $$
    \underbrace{(1 + i_t)}_{\text{Deposit Rate}} Y_{h,t} \geq \underbrace{[e_t + (1 - e_t)\phi]}_{\text{Prob. of Repay}}\underbrace{(1+r_t)}_{\text{Risky Rate}}\underbrace{(P_tC_{h,t} + P_t^X I_{h,t})}_{\text{Loan}}
    $$


}

\frame{\frametitle{Banks}
	
	Households have to pay in advance purchases of consumption services ($C_{h,t}$) and durable goods ($I_{h,t}$).
	
	
	
	Banks finance household purchases at interest rate $1 + r_t$ which satisfies the following zero-profit condition
	\begin{eqnarray*}
	1 + r_t = (1 + i_t) \frac{1 + (1 - e_t)\phi \Phi}{e_t + (1 - e_t)\phi}
	\end{eqnarray*}
	where risk premium is
		\begin{eqnarray}
	1 + r_t^p = \frac{1 + (1 - e_t)\phi \Phi}{e_t + (1 - e_t)\phi}
	\end{eqnarray}
	
	
	
}

\frame{\frametitle{Firms}


Intermediate firm $k$ produces consumption services as follows,
$$
C_{k,t} = s[X_{k,t} + \theta F(e_{k,t})], \ \ \ s > 0
$$
where $\theta$ is exogenous productivity. 

\

Moreover, the market for intermediate services is subject to sticky prices \`a la Calvo (1983).

\

Accordingly, final goods sector is competitive and combine $k$-specific consumption of services according to a Dixit-Stiglitz aggregator. 
	
	
}

\frame{\frametitle{Central Bank and Equilibrium}
	
	
	To close the model, central bank determine the risk-free rate $i_t$ according to
	$$
	1 + i_t = \Theta E_t [e_{t+1}^{\varphi_e} (1 + \pi_{t+1})]
	$$
	
	\
	
	Equilibrium is defined as
		\begin{eqnarray}
	X_{t+1} = (1-\delta)X_t + \psi \theta F(e_t)
	\end{eqnarray}
			\begin{eqnarray}
			\begin{aligned}
		 & U'\{s[X_t + \theta F(e_t)]  - \gamma s [X_{t-1} + \theta F(e_{t-1})] \} \\
		  = \ &  (1 + (1 - e_t)\phi \Phi) \beta \frac{\zeta_t}{\zeta_{t-1}}\Theta \\
		   \times \ & E_t \big[  \{s[X_t + \theta F(e_t)]  - \gamma s [X_{t-1} + \theta F(e_{t-1})] \}  e_{t+1}^{\varphi_e}\big]
			\end{aligned}
	\end{eqnarray}
}





\end{document}









\end{document}